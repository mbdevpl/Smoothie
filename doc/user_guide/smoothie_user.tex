\documentclass[a4paper,12pt]{article}

% very small margins (around 1.5cm)
% disable package geometry if you use this
\usepackage{fullpage}

% some mathematical symbols
\usepackage{amsmath}

% mathematical symbols
% conflicts with package program
\usepackage{amssymb}

% metadata for PDF file, also hyperlinks via \href{link}{description}
\usepackage{hyperref}
% clickable table of contents, package above also needed
\hypersetup{
    colorlinks,
    citecolor=black,
    filecolor=black,
    linkcolor=black,
    urlcolor=black
}

% use \includegraphics[scale=1.00]{file.jpg} for images
\usepackage{graphicx}

% for proper placement of floating figures
\usepackage{float}

\usepackage{multirow}
%\usepackage{pseudocode}
%\usepackage{booktabs}
%\usepackage{color}
%\usepackage{algorithmic} %pseudocode
%\usepackage{algorithm} %pseudocode as whole procedures
%\usepackage{program} %more sophisticated pseudocode

% use \begin{comment} ... \end{comment} for multiline comments
\usepackage{comment}

% use \begin{minted}[mathescape,linenos,numbersep=5pt,gobble=0,framesep=2mm]{c++}
\usepackage{minted} 

\begin{document}

\title{Smoothie -- user's guide}
\date{January 28, 2014}
\author{Mateusz Bysiek, Computer Science A/S, MiNI, WUT}
\maketitle

\section{Main menu}

Below, descriptions of what menu options do (only when the names are not self-explanatory):

\begin{itemize}

  \item Application

  \begin{itemize}

    \item About

    \item Exit 

  \end{itemize}

  \item Mesh

  \begin{itemize}

    \item New -- open a creator for creating new meshes

    \item Load -- load a mesh from XML file

    \item Load example -- load a predefined mesh from a list of available examples 

    \item Save -- save current mesh to a XML file

  \end{itemize}

  \item Mode -- change how the vizualization area behaves

  \begin{itemize}

    \item Mesh overview mode -- default mode

    \item Mesh editing mode -- allows modifications of existing points

    \item Colour points by boundary/free status -- points that have boundary condition enabled show in white, other
    points show in grey

    \item Colour points by value -- the points closest to minimum value are blue, then cyan, green, yellow and orange,
    and the points closest to maximum value are red

    \item Show only mesh

    \item Fill elements with texture

    \item Show all labels

    \item Hide all labels

  \end{itemize}

  \item FEM

  \begin{itemize}

    \item Start -- starts the computation for current mesh, blocks the UI until computation is completed

    \item Matrices -- opens a message box with data taken from matrices used in calculation  

  \end{itemize}

\end{itemize}

\newpage

\section{File format}

Smoothie is able to save and load text files in a custom-defined XML. The naming and structure of tags is made in hope
to be straightforward. Even if I hope that it is pretty intuitive, still below is the specific definition:

\begin{itemize}

  \item \emph{mesh} is a root tag. It defines domain of the mesh (starting and ending coordinates of each of 3
  dimensions). It contains definition of the type of the mesh (triangular or rectangular), as well as width and height
  which serve only as guides of how many points and elements the program can expect from the file. It also has
  \emph{name} that serves just as a descriptive name for the mesh.

  \item \emph{vertices} contains all vertices (a.k.a. points)

  \item \emph{elements} contains all elements (a.k.a. polygons)

  \item \emph{vertex} has a unique \emph{id}, that is then referenced in \emph{elements}, and it also has three coordinates
  \emph{x}, \emph{y} and \emph{z}, all of which should be (but don't have to) be in the domain. Moreover it has a
  property \emph{boundary} -- if it is 1 then this point is boundary, if it is 0 then it is not.

  \item \emph{element} has only one property, \emph{vertices} that is a list of ids of vertices, ids are separated by
  colons.

\end{itemize}

Example file:

\begin{minted}[linenos,numbersep=5pt]{xml}
<?xml version="1.0"?>
<mesh width="3" height="3" domain_start_x="-1.000000"
    domain_start_y="-1.000000" domain_start_z="0.000000"
    domain_end_x="1.000000" domain_end_y="1.000000"
    domain_end_z="2.000000" name="some 3 by 3 mesh" type="rectangular">
  <vertices>
    <vertex id="0" x="-1.000000" y="-1.000000" z="1.500000" boundary="1"/>
    <vertex id="1" x="0.000000" y="-1.000000" z="1.500000" boundary="1"/>
    <vertex id="2" x="1.000000" y="-1.000000" z="1.500000" boundary="1"/>
    <vertex id="3" x="-1.000000" y="0.000000" z="0.000000" boundary="0"/>
    <vertex id="4" x="0.000000" y="0.000000" z="0.000000" boundary="0"/>
    <vertex id="5" x="1.000000" y="0.000000" z="0.000000" boundary="0"/>
    <vertex id="6" x="-1.000000" y="1.000000" z="0.500000" boundary="1"/>
    <vertex id="7" x="0.000000" y="1.000000" z="0.500000" boundary="1"/>
    <vertex id="8" x="1.000000" y="1.000000" z="0.500000" boundary="1"/>
  </vertices>
  <elements>
    <element vertices="0;1;4;3"/>
    <element vertices="1;2;5;4"/>
    <element vertices="3;4;7;6"/>
    <element vertices="4;5;8;7"/>
  </elements>
</mesh>
\end{minted}

\newpage

\section{Visualisation area}

\begin{itemize}

  \item In ``overview mode'' use dragging with left mouse button to change the point of view.

  \item In ``editing mode'' use right mouse button click to edit data for a given point of the mesh.

\end{itemize}

There is a circle around the mesh, this circle serves as a guidance for where the lowest point (as defined by domain) of
the mesh is.

The visualisation area is also present in several other windows, but in those it is not interactive -- in mesh creation
window it serves just as a preview of what will be created, and in examples window it shows what examples exactly can be
loaded from there.

It is important to remark that labels are quite consuming when it comes to processing power, and also they tend to make
the mesh unreadable if mesh contains many elements. These are two good reasons to disable labels if you are going to
work with meshes of size 30 by 30 or even larger.

\section{Feedback from you}

Please send my any suggestions you have about the project, as I am not planning to abandon it. Main reason is that I am
using solutions developed here in my other open-source projects. Also, I welcome any contributions directly to the source
code. If you'd like to join the project as a contributor, also please send me an email. 

My address: \href{mailto:mb at mbdev.pl}{mb at mbdev.pl}  

\end{document}
